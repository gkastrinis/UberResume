\documentclass[a4paper]{resume}

\usepackage{bookmark}
\usepackage{graphicx}
\usepackage{paralist}
\usepackage{multirow}
\usepackage{marvosym}
\usepackage[LGR, T1]{fontenc}
\usepackage[english,greek]{babel}
\usepackage[utf8x]{inputenc}

% Modify document margins
\usepackage[left=0.75in,top=0.6in,right=0.75in,bottom=0.6in]{geometry}
\usepackage{hyperref}
\hypersetup{
  colorlinks=true,
  urlcolor=blue
}

\usepackage[
backend=bibtex,
isbn=false,
url=false,
doi=false,
abbreviate=false,
giveninits=true
]{biblatex}
\addbibresource{bib/resume.bib}
\addbibresource{bib/short-proceedings.bib}

% Logo Commands
\newcommand\githublogo{\raisebox{-1pt}{\includegraphics[height=9pt]{logos/github.pdf}}\ }
\newcommand\linkedinlogo{\raisebox{-1pt}{\includegraphics[height=9pt]{logos/linkedin.pdf}}\ }
\newcommand\emaillogo{\raisebox{-1pt}{\Letter}\ }
\newcommand\homephonelogo{\raisebox{-1pt}{\Telefon}\ }
\newcommand\cellphonelogo{\raisebox{-1pt}{\Mobilefone}\ }
\newcommand*{\github}[1]{\githublogo\href{https://#1}{#1}}
\newcommand*{\linkedin}[1]{\linkedinlogo\href{https://#1}{#1}}
\newcommand*{\email}[1]{\emaillogo\href{mailto:#1}{\nolinkurl{#1}}}
\newcommand*{\homephone}[1]{\homephonelogo{#1}}
\newcommand*{\cellphone}[1]{\cellphonelogo{#1}}

\newcommand{\tl}{\textlatin}
%\newcommand{\greekVersion}{}
\newcommand{\dual}[2]{\ifcsname greekVersion\endcsname #2\else\textlatin{#1}\fi}

\newcommand{\diFull}{\dual
{Ethnikon kai Kapodistriakon Panepistimion Athinon / University of Athens}
{Εθνικόν και Καποδιστριακόν Πανεπιστήμιον Αθηνών / Πανεπιστήμιο Αθηνών} \\
\dual
{Department of Informatics \& Telecommunications}
{Τμήμα Πληροφορικής \& Τηλεπικοινωνιών}}

%-----------------------
%   HEADER SECTION
%-----------------------

\name{\dual{Georgios ``George'' Kastrinis}{Γιώργος Καστρίνης} \diamond\;Curriculum Vit\ae}

\address{%\dual{X St.}{Δρόμος Χ} \\
%\dual{Argiroupoli (Athens 16452), Attica, Greece}{Αργυρούπολη (Αθήνα 16452), Αττική}
\hfill{\tl{\email{gkastrinis@di.uoa.gr}}}}

\address{%\homephone{+30 210 0000000} \hspace{0.2cm}
%\cellphone{+30 697 000 0000}
\hfill{\tl{\github{github.com/gkastrinis}}}}

\address{\hfill{\tl{\linkedin{gr.linkedin.com/in/george-kastrinis-828b4743}}}}

%%%%%%%%%%%%%%%%%%%%%%%%%%%%%%%%%%%%%%%%%%%%%%%%%%

\begin{document}
\newcommand{\mytilde}{\raise.17ex\hbox{$\scriptstyle\mathtt{\sim}$}}
\vspace{-1em}

%------------------------------
%     EDUCATION SECTION
%------------------------------

\begin{rSection}{\dual{Education}{Εκπαιδευση}}

\dual
{{\bf Doctor of Philosophy -- PhD,} Static Program Analysis \hfill {2012 -- (expected) Spring 2019}}
{{\bf Διδακτορικό -- \tl{PhD},} Στατική Ανάλυση Προγραμμάτων \hfill {2012 -- (εκτιμάται) Άνοιξη 2019}} \\
\diFull \\
\dual{Advisor: Prof.~Yannis Smaragdakis}{Επιβλέπων Καθηγητής: Γιάννης Σμαραγδάκης}
(\href{mailto:smaragd@di.uoa.gr}{\tl{\nolinkurl{smaragd@di.uoa.gr}}}) %\\
%\dual
%{Thesis: \emph{Scalable Static Pointer Analysis with High-Confidence Results}}
%{Διατριβή: \emph{Αποδοτική Στατική Ανάλυση Δεικτών με Αξιόπιστα Αποτελέσματα}}

\dual
{{\bf Master of Science,} Advanced Information Systems \hfill {2009 -- 2012}}
{{\bf Μεταπτυχιακό,} Προηγμένα Πληροφοριακά Συστήματα \hfill {2009 -- 2012}} \\
\diFull \\
\dual
{Grade: $8.42 / 10$ ($\sim$ 3\% of graduating class)}
{Μέσος Όρος: $8.42 / 10$ ($\sim$ 3\% της χρονιάς αποφοίτησης)}

\dual
{{\bf Bachelor of Science,} Computer Science \hfill {2005 -- 2009}}
{{\bf Πτυχίο,} Επιστήμη Υπολογιστών \hfill {2005 -- 2009}} \\
\diFull \\
\dual
{Grade: $8.33 / 10$ ($\sim$ 5\% of graduating class)}
{Μέσος Όρος: $8.33 / 10$ ($\sim$ 5\% της χρονιάς αποφοίτησης)}

\end{rSection}

%------------------------------
%   WORK EXPERIENCE SECTION
%------------------------------

\begin{rSection}{\dual{Experience}{Επαγγελματικη Εμπειρια}}

\begin{rSubsection}
  {\dual{University of Athens}{Πανεπιστήμιο ΑΘηνών}}
  {\dual{January 2011 -- present}{Ιανουάριος 2011 -- παρόν}}
  {\dual{Software Engineer, PLAST Lab}{Προγραμματιστής / Ερευνητής, \tl{PLAST Lab}}}
  {\dual{Athens, Greece}{Αθήνα, Ελλάδα}}
  \itemsep -0.5em
\item[] \emph{\dual{Supervisor}{Επιβλέπων}}: \dual{Prof.~Yannis Smaragdakis}{Καθηγητής~Γιαννης Σμαραγδάκης}
(\href{mailto:smaragd@di.uoa.gr}{\tl{\nolinkurl{smaragd@di.uoa.gr}}})
\item[] \dual
{\emph{Description}: Contributed to the development of the group's static analyses tools, more specifically the \textsc{Doop} framework, including implementing novel analyses as well as integrating the available analysis tools with a web application with the aim of showing static analysis results on the source code of the program.}
{\emph{Περιγραφή}: Συμβολή στην ανάπτυξη των στατικών εργαλείων της ομάδας, και πιο συγκεκριμένα στο εργαλείο \tl{\textsc{Doop}}. Ανάπτυξη νέων αναλύσεων καθώς και σύνδεση των αποτελεσμάτων υπαρχόντων εργαλείων με διαδικτυακή εφαρμογή με σκοπό την παρουσίαση τους παράλληλα με τον πηγαίο κώδικα του προγράμματος.}
\end{rSubsection}

%------------------------------------------------

\begin{rSubsection}
  {\tl{Microsoft Research}}
  {\dual{July 2014 -- October 2014}{Ιούλιος 2014 -- Οκτώβριος 2014}}
  {\dual{Software Engineer Intern}{Πρακτική - Προγραμματιστής}}
  {\tl{Redmond, WA, USA}}
  \itemsep -0.5em
\item[] \emph{\dual{Supervisor}{Επιβλέπων}}: \tl{Ben Livshits}
(\href{mailto:livshits@microsoft.com}{\tl{\nolinkurl{livshits@microsoft.com}}})
\item[] \dual
{\emph{Description}: Contributed to the development of a platform utilizing Amazon's Mechanical Turk in order to offer a programmable API for creating surveys while taking into consideration a certain financial budget and/or time constraints.}
{\emph{Περιγραφή}: Συμβολή στην ανάπτυξη πλατφόρμας, χρησιμοποιώντας και την υπηρεσία της \tl{Amazon - Mechanical Turk}, με σκοπό την υλοποίηση βιβλιοθήκης / προγραμματιστικού \tl{API} για τη δημιουργία διαδικτυακών δημοσκοπήσεων / ερευνών, παίρνοντας υπόψιν χρηματικούς και χρονικούς περιορισμούς.}
\end{rSubsection}

%------------------------------------------------

\begin{rSubsection}
  {\tl{LogicBlox, Inc}}
  {\dual{May 2013 -- September 2013}{Μάιος 2013 -- Σεπτέμβριος 2013}}
  {\dual{Software Engineer Intern}{Πρακτική - Προγραμματιστής}}
  {\tl{Atlanta, GA, USA}}
  \itemsep -0.5em
\item[] \emph{\dual{Supervisor}{Επιβλέπων}}: \tl{Martin Bravenboer}
(\href{mailto:martin.bravenboer@logicblox.com}{\tl{\nolinkurl{martin.bravenboer@logicblox.com}}})}
\item[] \dual
{\emph{Description}: Contributed to the development of a tool to integrate \textsc{CLANG} information to a static analysis framework expressed in Datalog.}
{\emph{Περιγραφή}: Συμβολή στην ανάπτυξη εργαλείου για τη σύνδεση πληροφορίας από το εργαλέιο \tl{\textsc{CLANG}} με μία στατική ανάλυση προγραμμάτων εκφρασμένη στη γλώσσα προγραμματισμού \tl{Datalog}.}
\end{rSubsection}

%------------------------------------------------

\begin{rSubsection}
  {\dual{Network Operations Center, University of Athens}{Κέντρο Διαχείρησης Δικτύου (\tl{NOC}), Πανεπιστήμιο Αθηνών}}
  {\dual{June 2009 -- January 2011}{Ιούνιος 2009 -- Ιανουάριος 2011}}
  {\dual{Software Engineer}{Προγραμματιστής / Ερευνητής}}
  {\dual{Athens, Greece}{Αθήνα, Ελλάδα}}
  \itemsep -0.5em
\item[] \dual
{\emph{Description}: Developed and maintained various web services and platforms offered to University users by the Network Operations Center.}
{\emph{Περιγραφή}: Ανάπτυξη και συντήρηση διάφορων υπηρεσιών και εργαλείων που προσφέρονται απο το Κέντρο στους χρήστες του Πανεπιστημίου.}
\end{rSubsection}

\end{rSection}

%\newpage

\begin{rSection}{\dual{Languages and Certificates}{Ξενες Γλωσσες}}
  \dual{Greek (native), English (fluent), French (basic)}
  {Ελληνικά (μητρική), Αγγλικά (υψηλό επίπεδο), Γαλλικά (βασικό επίπεδο)} \\
  \selectlanguage{english}Certificate of Proficiency in English, University of Michigan (USA) \& University of Cambridge (UK) \\
  DEFL A1-A6
  \selectlanguage{greek}
\end{rSection}

%----------------------------------
%   TECHNICAL STRENGTHS SECTION
%----------------------------------

\begin{rSection}{\dual{Technical Skills}{Τεχνικες Δεξιοτητες}}

{\renewcommand{\arraystretch}{1.3}
\begin{tabular}{ @{} >{\bfseries}l @{\hspace{6ex}} l }

\dual{Programming Languages}{Γλώσσες Προγραμματισμού}
   & \tl{C/C++, C\#, Java, Groovy, Python, Prolog, Datalog, Haskell} \\

\dual{Build Tools}{\tl{Build Tools}}
   & \tl{GNU Make, ANT, Gradle} \\

\dual{Version Control}{\tl{Version Control}}
   & \tl{Git, Mercurial, Subversion} \\

\dual{Databases \& ORM}{Βάσεις Δεδομένων}
   & \tl{SQL, MySQL, LogicBlox, SPARQL, OWL 2, LDAP, LINQ} \\

\dual{Web Technologies}{Τεχνολογίες Διαδικτύου}
   & \tl{HTML, CSS, Javascript, PHP, NodeJS} \\

\dual{Misc}{Διάφορα}
   & \tl{ANTLR 4, Vim, IntelliJ IDEA, Eclipse, GDB} \\
\end{tabular}}
\end{rSection}

%---------------------------
%   PUBLICATIONS SECTION
%---------------------------

\begin{rSection}{\dual{Publications}{Δημοσιευσεις}}
  \selectlanguage{english}
  \begin{rSubsection}{}{}{}{}
    \itemsep -0.5em
  \item \fullcite{soundmay}.
  \item \fullcite{stringcoloring}.
  \item \fullcite{proxies}.
  \item \fullcite{mustaliasdata}.
  \item \fullcite{mustaliasdatalog}.
  \item \fullcite{reflection}.
  \item \fullcite{introspective}.
  \item \fullcite{setbased}.
  \item \fullcite{hybrid}.
  \item \fullcite{exceptions}.
  \item \fullcite{localsearch}.
  \end{rSubsection}
  \selectlanguage{greek}
\end{rSection}

%-------------------------------------------
%   EXTRA-CURRICULAR ACTIVITIES SECTION
%-------------------------------------------

\begin{rSection}{\dual{Activities}{Δραστηριοτητες}}
  {\bf\dual{Teaching Assistant}{Υποστήριξη Μαθήματος}} \hfill {2009 -- \dual{present}{παρόν}} \\
  \dual{University of Athens - Dept. of Informatics \& Telecommunications}
  {Πανεπιστήμιο ΑΘηνών - Τμήμα Πληροφορικής \& Τηλεπικοινωνιών} \\
  \dual{Courses: \emph{OOP, Advanced OOP, Systems Programming, Introduction to Programming}}
  {Μαθήματα: \emph{\tl{OOP}, Προηγμένα Θέματα \tl{OOP}, Προγρ. Συστήματος, Εισαγωγή στον Προγραμματισμό}} \\
  {\bf\tl{Google PhD Student Summit on Compilers}} \hfill {\dual{December}{Δεκέμβριος} 2016} \\
  {\bf\tl{PLDI '15 Student Volunteer}} \hfill {\dual{June}{Ιούνιος} 2015} \\
\end{rSection}

\end{document}